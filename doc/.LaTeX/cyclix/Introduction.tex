
%\begin{frame}[allowframebreaks]{\textbf{Comments}} \label{Introduction}
%The code will fail with the following options and the related input options are listed.
%\begin{itemize}
%  \item Polarized calculation: \texttt{SPIN\_TYP}.
%  \item K-point calculation: \texttt{KPOINT\_GRID}, \texttt{KPOINT\_SHIFT}.
%  \item Dirichlet boundary condition in any direction: \texttt{BC}
%  \item Define number of states/orbitals: \texttt{NSTATES}
%  \item Hybrid functionals: \texttt{EXCHANGE\_CORRELATION}
%  \item Print eigenvalues into file: \texttt{PRINT\_EIGEN}
%\end{itemize}
%
%\end{frame}


  \begin{frame}[allowframebreaks]{\textbf{Contributors}} \label{Contributors}
  %\begin{itemize}
      %\item U.S. Department of Energy, Office of Science: DE-SC0019410
      %\item U.S. National Science Foundation: 1333500 and 1553212
  %\end{itemize}

	\begin{itemize}
	\item \textbf{Phanish Suryanarayana} (PI)
	\begin{itemize}
 	    \item \textbf{Abhiraj Sharma}: Code infrastructure, CheFSI, PBE, Energy, Force, Stress, atomic \& cell relaxation \\   
 	    \item \textbf{Qimen Xu}: Code infrastructure \\  
        \item \textbf{Xin Jing}: Code infrastructure, Spin, SOC \\  
        \item \textbf{Arpit Bhardwaj}: Spin, SOC \\  
% 	    \item \textbf{Boqin Zhang}: vdW-DF, DFT-D3, meta-GGA (SCAN) \\  
% 	    \item \textbf{Shashikant Kumar}: Testing framework, NLCC \\  
%            \item \textbf{Mostafa Faghih Shojaei}: SPMS table of pseudopotentials \\  
 	\end{itemize}
%    \item \textbf{John E. Pask} (co-PI)
%    \item \textbf{Edmond Chow} (co-PI)
%    \begin{itemize}
%        \item \textbf{Hua Huang}: Subspace eigensolver, DP
%        \item \textbf{Lucas Erlandson}: Subspace eigensolver, DP
%    \end{itemize}
%    \item \textbf{Andrew J. Medford} (co-PI)
%    \begin{itemize}
%        \item \textbf{Benjamin Comer}: Code testing, Initial testing framework
%        \item \textbf{Sushree Jagriti Sahoo}: Code testing
%    \end{itemize}
 	\end{itemize}
  
\end{frame}


\begin{frame}[allowframebreaks]{\textbf{Citation}} \label{Citation}
If you publish work using/regarding SPARC-cyclix, please cite the following article, in addition to SPARC citations:
\begin{itemize}
    \item \url{https://doi.org/10.1103/PhysRevB.103.035101}, \url{https://doi.org/10.1103/PhysRevB.100.125143} (initial developments, extended systems), \url{https://doi.org/10.1016/j.jmps.2016.08.007} (initial developments, isolated systems)
%    \item Additional references for initial developments: \url{https://doi.org/10.1016/j.jmps.2016.08.007}, \url{https://doi.org/10.1103/PhysRevB.100.125143}
%    \item Non-orthogonal systems: \url{https://doi.org/10.1016/j.cplett.2018.04.018}
%    \item Linear solvers: \url{https://doi.org/10.1016/j.cpc.2018.07.007},    \url{https://doi.org/10.1016/j.jcp.2015.11.018}
%    \item Stress tensor/pressure: \url{https://doi.org/10.1063/1.5057355}
%    \item Atomic forces: \url{https://doi.org/10.1016/j.cpc.2016.09.020}, \url{https://doi.org/10.1016/j.cpc.2017.02.019}
%    \item Mixing: \url{https://doi.org/10.1016/j.cplett.2016.01.033}, \url{https://doi.org/10.1016/j.cplett.2015.06.029}, \url{https://doi.org/10.1016/j.cplett.2019.136983}
%    \item SPMS pseudopotentials: \url{https://doi.org/10.1016/j.cpc.2022.108594}
\end{itemize}
\end{frame}


\begin{frame}[allowframebreaks]{\textbf{Acknowledgements}} \label{Acknowledgements}
\begin{itemize}
    \item \textbf{U.S. National Science Foundation (NSF): 1553212 
        } \\
\end{itemize}
\end{frame}


\begin{frame}[allowframebreaks]{\textbf{Input file options}} \label{Index}
\vspace{-2mm}
Input file options for SPARC-cyclix, in addition to SPARC:
 \begin{block}{Cyclix}
\hyperlink{TWIST_ANGLE}{\texttt{TWIST\_ANGLE}} $\vert$ 
\hyperlink{BC}{\texttt{BC}} $\vert$ 
\hyperlink{CELL}{\texttt{CELL}}  $\vert$ 
\hyperlink{COORD}{\texttt{COORD}}  $\vert$ 
\hyperlink{COORD_FRAC}{\texttt{COORD\_FRAC}}  $\vert$ 
\hyperlink{EXCHANGE_CORRELATION}{\texttt{EXCHANGE\_CORRELATION}} $\vert$ 
\hyperlink{KPOINT_GRID}{\texttt{KPOINT\_GRID}} $\vert$ 

%\hyperlink{SPIN_TYP}{\texttt{SPIN\_TYP}}  $\vert$ 
%\hyperlink{SPIN}{\texttt{SPIN}}  $\vert$ 
\end{block}
\end{frame}









