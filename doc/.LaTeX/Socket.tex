
%%%%%%%%%%%%%%%%%%%%%%%%%%%%%%%%%%%%%%%%%%%%%%%%%%%%%%%%%%%%%%%%%%%%%%%%%%%%%%%%%%%%%%%%%%%%%
\begin{frame}[allowframebreaks,c]{} \label{Socket}

\begin{center}
\Huge \textbf{Socket communication in SPARC}
\end{center}

\end{frame}
%%%%%%%%%%%%%%%%%%%%%%%%%%%%%%%%%%%%%%%%%%%%%%%%%%%%%%%%%%%%%%%%%%%%%%%%%%%%%%%%%%%%%%%%%%%%%

%%%%%%%%%%%%%%%%%%%%%%%%%%%%%%%%%%%%%%%%%%%%%%%%%%%%%%%%%%%%%%%%%%%%%%%%%%%%%%%%%%%%%%%%%%%%%
\begin{frame}[allowframebreaks]{\textbf{Brief introduction}} \label{SOCKET_INTRO}
  
  The socket communication layer allows SPARC to be controlled by a socket server that is compatible with
  \href{https://ipi-code.org/i-pi/introduction.html}{i-PI} protocol. The SPARC source code should be compiled with the
  \texttt{USE\_SOCKET=1} option. 

  To start a SPARC program with socket interface, use either option:
  \begin{itemize}
  \item Option 1: Specify socket address in command line
\begin{itemize}
    \item To start an INET socket, use\\
    \texttt{\$ mpirun -n 8 ./lib/sparc -socket localhost:12345 -name filename}

    \item To start a UNIX socket, use\\
    \texttt{\$ mpirun -n 8 ./lib/sparc -socket /tmp/sparc.socket:unix -name filename}
  \end{itemize}
    
  \item Option 2: Provide \texttt{.inpt} parameters

    Visit the following pages for parameter specifications.
    Any parameter \texttt{SPARC\_*} in the \texttt{.inpt} file
    will overwrite the settings from command line.
  \end{itemize}

\end{frame}
%%%%%%%%%%%%%%%%%%%%%%%%%%%%%%%%%%%%%%%%%%%%%%%%%%%%%%%%%%%%%%%%%%%%%%%%%%%%%%%%%%%%%%%%%%%%%

%%%%%%%%%%%%%%%%%%%%%%%%%%%%%%%%%%%%%%%%%%%%%%%%%%%%%%%%%%%%%%%%%%%%%%%%%%%%%%%%%%%%%%%%%%%%%
\begin{frame}[allowframebreaks]{\textbf{Output files}} \label{SOCKET_OUTPUT}
  
  The following files will be generated in the socket mode and may slightly
  differ from standard SPARC outputs:
  \begin{itemize}
  \item \texttt{.out}: All the SCF steps will be written to  \texttt{.out} file, similar to a relaxation / MD calculation.
  \item \texttt{.static}: It is concatenated from all single point
    steps.  In addition to normal \texttt{.static} output, the lattice
    information are also recorded.
 \end{itemize}

\end{frame}
%%%%%%%%%%%%%%%%%%%%%%%%%%%%%%%%%%%%%%%%%%%%%%%%%%%%%%%%%%%%%%%%%%%%%%%%%%%%%%%%%%%%%%%%%%%%%

%%%%%%%%%%%%%%%%%%%%%%%%%%%%%%%%%%%%%%%%%%%%%%%%%%%%%%%%%%%%%%%%%%%%%%%%%%%%%%%%%%%%%%%%%%%%%
\begin{frame}[allowframebreaks]{\texttt{SOCKET\_FLAG}} \label{SOCKET_FLAG}
\vspace*{-12pt}
\begin{columns}
\column{0.4\linewidth}
\begin{block}{Type}
Integer
\end{block}

\begin{block}{Default}
0
\end{block}

\column{0.4\linewidth}
\begin{block}{Unit}
No unit
\end{block}

\begin{block}{Example}
\texttt{SOCKET\_FLAG}: 1
\end{block}
\end{columns}

\begin{block}{Description}
  Flag for starting the socket communication layer. It is equivalent to the \texttt{-socket} switch of command line.

  Setting \texttt{SOCKET\_FLAG: 1} will disable \texttt{MD\_FLAG} and \texttt{RELAX\_FLAG}.
  
\end{block}


\end{frame}
%%%%%%%%%%%%%%%%%%%%%%%%%%%%%%%%%%%%%%%%%%%%%%%%%%%%%%%%%%%%%%%%%%%%%%%%%%%%%%%%%%%%%%%%%%%%%

%%%%%%%%%%%%%%%%%%%%%%%%%%%%%%%%%%%%%%%%%%%%%%%%%%%%%%%%%%%%%%%%%%%%%%%%%%%%%%%%%%%%%%%%%%%%%
\begin{frame}[allowframebreaks]{\texttt{SOCKET\_HOST}} \label{SOCKET_HOST}
\vspace*{-12pt}
\begin{columns}
\column{0.4\linewidth}
\begin{block}{Type}
String
\end{block}

\begin{block}{Default}
localhost
\end{block}

\column{0.4\linewidth}
\begin{block}{Unit}
No unit
\end{block}

\begin{block}{Example}
\texttt{SOCKET\_HOST}: 127.0.0.1
\end{block}
\end{columns}

\begin{block}{Description}
  Host name of the socket address that SPARC listens to.  If it's an
  INET socket, it is the address of the interface.  For a UNIX socket,
  it is the filename of the socket file
  (e.g. \texttt{/tmp/sparc.socket}).
\end{block}

\end{frame}
%%%%%%%%%%%%%%%%%%%%%%%%%%%%%%%%%%%%%%%%%%%%%%%%%%%%%%%%%%%%%%%%%%%%%%%%%%%%%%%%%%%%%%%%%%%%%

%%%%%%%%%%%%%%%%%%%%%%%%%%%%%%%%%%%%%%%%%%%%%%%%%%%%%%%%%%%%%%%%%%%%%%%%%%%%%%%%%%%%%%%%%%%%%
\begin{frame}[allowframebreaks]{\texttt{SOCKET\_PORT}} \label{SOCKET_PORT}
\vspace*{-12pt}
\begin{columns}
\column{0.4\linewidth}
\begin{block}{Type}
Integer
\end{block}

\begin{block}{Default}
None
\end{block}

\column{0.4\linewidth}
\begin{block}{Unit}
No unit
\end{block}

\begin{block}{Example}
\texttt{SOCKET\_PORT}: 12345
\end{block}
\end{columns}

\begin{block}{Description}
  When SPARC connects to an INET socket server, it is the port number. The \texttt{SOCKET\_PORT} has no effect for a UNIX socket.
\end{block}

\end{frame}
%%%%%%%%%%%%%%%%%%%%%%%%%%%%%%%%%%%%%%%%%%%%%%%%%%%%%%%%%%%%%%%%%%%%%%%%%%%%%%%%%%%%%%%%%%%%%

%%%%%%%%%%%%%%%%%%%%%%%%%%%%%%%%%%%%%%%%%%%%%%%%%%%%%%%%%%%%%%%%%%%%%%%%%%%%%%%%%%%%%%%%%%%%%
\begin{frame}[allowframebreaks]{\texttt{SOCKET\_INET}} \label{SOCKET_INET}
\vspace*{-12pt}
\begin{columns}
\column{0.4\linewidth}
\begin{block}{Type}
Integer
\end{block}

\begin{block}{Default}
None
\end{block}

\column{0.4\linewidth}
\begin{block}{Unit}
No unit
\end{block}

\begin{block}{Example}
\texttt{SOCKET\_INET}: 0
\end{block}
\end{columns}

\begin{block}{Description}
  1 for INET socket, 0 for UNIX socket. If no set in \texttt{.inpt}
  file, its value is determined by the \texttt{-socket address:port}
  command line switch.
\end{block}

\end{frame}
%%%%%%%%%%%%%%%%%%%%%%%%%%%%%%%%%%%%%%%%%%%%%%%%%%%%%%%%%%%%%%%%%%%%%%%%%%%%%%%%%%%%%%%%%%%%%

%%%%%%%%%%%%%%%%%%%%%%%%%%%%%%%%%%%%%%%%%%%%%%%%%%%%%%%%%%%%%%%%%%%%%%%%%%%%%%%%%%%%%%%%%%%%%
\begin{frame}[allowframebreaks]{\texttt{SOCKET\_MAX\_NITER}} \label{SOCKET_MAX_NITER}
\vspace*{-12pt}
\begin{columns}
\column{0.4\linewidth}
\begin{block}{Type}
Integer
\end{block}

\begin{block}{Default}
10000
\end{block}

\column{0.4\linewidth}
\begin{block}{Unit}
No unit
\end{block}

\begin{block}{Example}
\texttt{SOCKET\_MAX\_NITER}: 10000
\end{block}
\end{columns}

\begin{block}{Description}
  Maximum number of ionic SCF steps in the socket mode. As a socket
  client, SPARC will terminate after \texttt{SOCKET\_MAX\_NITER} steps
  are called.
\end{block}


\end{frame}
%%%%%%%%%%%%%%%%%%%%%%%%%%%%%%%%%%%%%%%%%%%%%%%%%%%%%%%%%%%%%%%%%%%%%%%%%%%%%%%%%%%%%%%%%%%%%